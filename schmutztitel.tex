% Schmutztitel wird einseitig gesetzt, um Layout
% nicht zu zerschreddern
% Wer den Schmutztitel nicht haben will, l\"{o}scht diesen Teil einfach
\KOMAoptions{twoside=false}
% Beginn des Schmutztitels als Teil der Titlepage-Umgebung
\begin{titlepage}%Schmutztitel; Grafiken mit Tikz erstellt; Signet eingebunden
\begin{tikzpicture}[remember picture,overlay,shift={(current page.south)}]


\node at (7.5,25.5) {\includegraphics[width = 0.32\textwidth]{logo1.jpg}};
\node at (-2.5,26.8) {\includegraphics[width = 0.85\textwidth]{THM_logo.eps}};

%\fill[left color=black!100, right color=black!100](-8,7.5) rectangle +(16,9);


\node [text width=13cm] at (0, 13.5)
{
\sffamily
\begin{flushleft}
    \Large
    \text{Bachelorarbeit}\\[3ex]
    \vspace{.5cm}
    \LARGE
\color{black}
\textbf{Weiterentwicklung und ionenoptische Simulation eines Stoßionisations-Flugzeitmassenspektrometers}\\[3ex]
\vspace{.5cm}
\Large
\textbf{Development and Ion-Optical Simulation of an Electron-Impact Ionization Time-of-Flight Mass Spectrometer}\\[.9ex]
\vspace{1cm}

\text{Lorenz Saalmann}\\[1ex]
\small\text{WS 24/25}\\[.9ex]
\end{flushleft}
};

\node [text width=8cm] at (10.7,2){\includegraphics[width = 0.45\textwidth]{PTRA_Logo_weiß.jpeg}};
\centering
\node [text width=8cm] at (3,24){\includegraphics[width = 0.32\textwidth]{Fachschaftslogo.png}};
\hspace{-.1cm}\node [text width=8cm] at (3,21){\includegraphics[width = 0.33\textwidth]{IPI.png}};

% Auskommentieren erzeugt Hilfslinien und Punkte auf Schmutztitelseite
% kann hilfreich für Anpassungen sein
%\draw [help lines] (-10,0) grid (10,30);
%\draw [fill=black] (0, 0) circle (0.1);
%\draw [fill=black] (0, 5) circle (0.1);
%\draw [fill=black] (0, 10) circle (0.1);
%\draw [fill=black] (0, 15) circle (0.1);
%\draw [fill=black] (0, 20) circle (0.1);
%\draw [fill=black] (0, 25) circle (0.1);
%\draw [fill=black] (5, 0) circle (0.1);
%\draw [fill=black] (5, 5) circle (0.1);
%\draw [fill=black] (5, 10) circle (0.1);
%\draw [fill=black] (5, 15) circle (0.1);
%\draw [fill=black] (5, 20) circle (0.1);
%\draw [fill=black] (5, 25) circle (0.1);
%\draw [fill=black] (-10, 0) circle (0.1);
%\draw [fill=black] (-10, 5) circle (0.1);
%\draw [fill=black] (-10, 10) circle (0.1);
%\draw [fill=black] (-10, 15) circle (0.1);
%\draw [fill=black] (-10, 20) circle (0.1);
%\draw [fill=black] (-10, 25) circle (0.1);
%\draw [fill=black] (-5, 0) circle (0.1);
%\draw [fill=black] (-5, 5) circle (0.1);
%\draw [fill=black] (-5, 10) circle (0.1);
%\draw [fill=black] (-5, 15) circle (0.1);
%\draw [fill=black] (-5, 20) circle (0.1);
%\draw [fill=black] (-5, 25) circle (0.1);
%\draw [fill=black] (10, 0) circle (0.1);
%\draw [fill=black] (10, 5) circle (0.1);
%\draw [fill=black] (10, 10) circle (0.1);
%\draw [fill=black] (10, 15) circle (0.1);
%\draw [fill=black] (10, 20) circle (0.1);
%\draw [fill=black] (10, 25) circle (0.1);

\end{tikzpicture}
\end{titlepage}
% Ende des Schmutztitels 