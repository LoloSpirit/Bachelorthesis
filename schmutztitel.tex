% Schmutztitel wird einseitig gesetzt, um Layout
% nicht zu zerschreddern
% Wer den Schmutztitel nicht haben will, l\"{o}scht diesen Teil einfach
\KOMAoptions{twoside=false}
% Beginn des Schmutztitels als Teil der Titlepage-Umgebung
\begin{titlepage}%Schmutztitel; Grafiken mit Tikz erstellt; Signet eingebunden
\begin{tikzpicture}[remember picture,overlay,shift={(current page.south)}]


\node at (7.5,25.5) {\includegraphics[width = 0.32\textwidth]{logo1.jpg}};
\node at (-2.5,26.8) {\includegraphics[width = 0.85\textwidth]{THM_logo.eps}};

\fill[left color=jlublau!100, right color=jlublau!100](-10,4) rectangle +(17.5,8);

\node [text width=14cm] at (-1.0, 8.4)
{
\sffamily
\color{white}
\begin{center}
\Large
\textbf{Bachelorarbeit}\\[1ex]
\LARGE
\textbf{Weiterentwicklung und ionenoptische Simulation eines Stoßionisations-Flugzeitmassenspektrometers}\\[1.2ex]
\Large
\textbf{Development and Ion-Optical Simulation of an Electron-Impact Ionization Time-of-Flight Mass Spectrometer}\\[1ex]
\textbf{Lorenz Saalmann}\\[1.2ex]
\Large
\text{Wintersemster 2024}
\end{center}
};

\node [text width=8cm] at (11,2){\includegraphics[width = 0.32\textwidth]{PTRA_logo.png}};

% Auskommentieren erzeugt Hilfslinien und Punkte auf Schmutztitelseite
% kann hilfreich für Anpassungen sein
%\draw [help lines] (-10,0) grid (10,30);
%\draw [fill=black] (0, 0) circle (0.1);
%\draw [fill=black] (0, 5) circle (0.1);
%\draw [fill=black] (0, 10) circle (0.1);
%\draw [fill=black] (0, 15) circle (0.1);
%\draw [fill=black] (0, 20) circle (0.1);
%\draw [fill=black] (0, 25) circle (0.1);
%\draw [fill=black] (5, 0) circle (0.1);
%\draw [fill=black] (5, 5) circle (0.1);
%\draw [fill=black] (5, 10) circle (0.1);
%\draw [fill=black] (5, 15) circle (0.1);
%\draw [fill=black] (5, 20) circle (0.1);
%\draw [fill=black] (5, 25) circle (0.1);
%\draw [fill=black] (-10, 0) circle (0.1);
%\draw [fill=black] (-10, 5) circle (0.1);
%\draw [fill=black] (-10, 10) circle (0.1);
%\draw [fill=black] (-10, 15) circle (0.1);
%\draw [fill=black] (-10, 20) circle (0.1);
%\draw [fill=black] (-10, 25) circle (0.1);
%\draw [fill=black] (-5, 0) circle (0.1);
%\draw [fill=black] (-5, 5) circle (0.1);
%\draw [fill=black] (-5, 10) circle (0.1);
%\draw [fill=black] (-5, 15) circle (0.1);
%\draw [fill=black] (-5, 20) circle (0.1);
%\draw [fill=black] (-5, 25) circle (0.1);
%\draw [fill=black] (10, 0) circle (0.1);
%\draw [fill=black] (10, 5) circle (0.1);
%\draw [fill=black] (10, 10) circle (0.1);
%\draw [fill=black] (10, 15) circle (0.1);
%\draw [fill=black] (10, 20) circle (0.1);
%\draw [fill=black] (10, 25) circle (0.1);

\end{tikzpicture}
\end{titlepage}
% Ende des Schmutztitels 